\documentclass{book}
\usepackage{tabularx}
\usepackage{booktabs}
\usepackage{graphicx}
\usepackage{listings}
\usepackage{hyperref}
\hypersetup{
    colorlinks,
    citecolor=black,
    filecolor=black,
    linkcolor=black,
    urlcolor=black
}
\graphicspath{ {./images/} }
\title{Manuel utilisateur de l'application Super Bowl}
\author{Benjamin BALET}
\date{Septembre 2023}
\begin{document}
\maketitle{}
\tableofcontents
\newpage


\chapter{Introduction}
\subsection{Comptes utilisateurs}

Les comptes utilisateurs sont listés dans le tableau \ref{tab:table1}. Ces comptes sont utilisables sur les
trois applications (web, mobile et bureautique).

\begin{table}
  \label{tab:table1}
    \begin{tabular}{l|l|l} \toprule
      \textbf{Login} & \textbf{Mot de passe} & \textbf{Rôle}\\ \hline
      admin@example.org & admin & Administrateur  \\ \hline
      commentateur@example.org & commentateur & Commentateur  \\ \hline
      visiteur@example.org & visiteur & Visiteur sans pari  \\ \hline
      user0 & user0@example.org  & Utilisateur avec au moins un pari  \\ \hline
      \ldots & \ldots  & \ldots  \\ \hline
      user20 & user20@example.org & Utilisateur avec au moins un pari  \\ \hline
    \end{tabular}
    \caption{Comptes utilisateurs}
\end{table}

Les comptes utilisateurs sont créés à l'aide d'une fixture Symfony. Les mots de passe sont cryptés avec 
l'algorithme bcrypt. Pour créer les comptes utilisateurs, exécutez la commande suivante :
\begin{lstlisting}[language=bash]
  $ php bin/console doctrine:fixtures:load --no-interaction
\end{lstlisting}


\subsection{Habilitations}

Une habilitation est un droit d'accès à une fonctionnalité de l'application. Les habilitations sont listées 
dans le tableau \ref{tab:table2}. On note que les administrateurs ont accès à toutes les fonctionnalités de
l'application. 

Certaines fonctionnalités sont publiques, c'est à dire qu'elles sont accessibles à tous les utilisateurs, même non connectés :
\begin{itemize}
  \item la page d'accueil
  \item la page de connexion
  \item la page d'inscription
  \item la page de visualisation des paris
  \item la page de visualisation des matchs et du détail de chaque match
\end{itemize}

\begin{table}
  \label{tab:table2}
    \begin{tabular}{l|c|c|c|c} \toprule
      \textbf{Compte} & \textbf{Parier} & \textbf{Commenter} & \textbf{Score} & \textbf{Administrer} \\ \hline
      Administrateur & X & X & X & X  \\ \hline
      commentateur & X & X & X &  \\ \hline
      Visiteur & X & & & \\ \hline
      Utilisateur & X & & & \\ \hline
    \end{tabular}
    \caption{Habilitation}
\end{table}


\chapter{Application Web}

\section{Connexion}

Utilisez un des comptes utilisateur listés dans le tableau \ref{tab:table1} pour vous connecter à l'application.

\chapter{Application Mobile}


\chapter{Application Bureautique}


\begin{appendix}
    \listoffigures
    \listoftables
  \end{appendix}
\end{document}
