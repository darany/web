\documentclass{book}
\usepackage[cyr]{aeguill}
\usepackage[francais]{babel}
\usepackage{tabularx}
\usepackage{graphicx}
\usepackage{float}
\usepackage{lscape}
\graphicspath{ {./images/} }
\title{Dossier de conception technique - application Super Bowl}
\author{Benjamin BALET}
\date{Septembre 2023}
\begin{document}
\maketitle{}
\tableofcontents
\newpage

\chapter{Introduction}

\section{Contexte}

Tous les ans a lieu un grand évènement, le Super Bowl. Cet évènement est une compétition de football américain
 qui est vue par des millions de personnes partout dans le monde. Stania, est une société mandatée par le 
 Super Bowl afin de pouvoir mettre en place des applications permettant de favoriser les paris sportifs ainsi 
 que la gestion des matchs. Stania est une start-up naissante passionnée de sport ! Ils ont construit plusieurs 
 applications pour d'autres événements, maisjamais pour un aussi gros client, c'est un enjeu très important pour eux.

Afin de pouvoir réussir le projet, ils ont fait une sélection drastique, ils ont voulu le ou la meilleur(e) 
de tous les postulants. Fort de votre expérience au sein du Bachelor Concepteur Développeur d'Application, 
vous avez candidaté et réussi(e) toutes les épreuves ! Vous faites partie de Stania désormais. Félicitations !
 Votre mission sera d'aider vos collaborateurs à développer les applications demandées par le Super Bowl !

\section{Cahier des charges}

Le besoin donné par le Super Bowl comporte trois applications décrites dans les paragraphes suivants.
Application Web :
\begin{itemize}
    \item	Possibilité d'effectuer des paris sportifs, ces paris sont configurables par l'administrateur
    \item	Les utilisateurs peuvent miser des montants sur l'équipe qu'ils souhaitent.
    \item	Le paiement a lieu sur place, au stade.
    \item	Les utilisateurs peuvent visualiser les matchs à venir et miser dessus à l'avance.
    \item	Quand un match a commencé, il n'est plus possible de miser.
\end{itemize}
Application mobile :
\begin{itemize}
    \item	Seuls les matchs où l'utilisateur a misé sur l'application web sont visibles.
    \item	Les matchs passés sont affichés (grisé) ainsi que ceux à venir.
    \item	Le match en cours est affiché de manière qu'on le remarque en premier.
    \item	Au clic, nous avons accès à tous les commentaires du match ainsi qu'au score en direct (après rafraîchissement). L'utilisateur doit pouvoir retrouver le montant de son pari et envers quelle équipe.
\end{itemize}
Application bureautique :
\begin{itemize}
    \item	Lancement du début du match
    \item	Écriture des commentaires pendant que le match se joue et mise à jour des scores quand un but est effectué.
    \item	Fermeture du match
Le but est de marquer un touch down avec ses trois applications ! (Un touch down signifie franchir la ligne d'en-but avec le ballon afin de marquer des points)
\end{itemize}

\subsection{Hyptohèses}

1. Le cahier des charges mentionne la nécessité de pouvoir créer des équipes et des joueurs, mais pas les autres opérations CRUD. 
Nous supposons nécessaire de pouvoir créer, modifier et supprimer des équipes et des joueurs.
Nous supposons toutefois qu'il n'est pas possible de supprimer un joueur appartenant à une équipe.
Nous supposons également qu'il n'est pas possible de supprimer une équipe dans les conditions suivantes :
\begin{itemize}
    \item L'équipe possède des joueurs
    \item L'équipe a participé à un match
    \item Un pari a été fait sur l'équipe
\end{itemize}

2. Le cahier des charges mentionne la nécessité de pouvoir créer des matchs (\textit{Affectation des équipes}), mais pas les autres opérations CRUD.
Nous supposons nécessaire de pouvoir créer, modifier et supprimer des matchs.
Nous supposons que la météo n'est modifiable que par l'administrateur.
Nous supposons toutefois qu'il n'est pas possible de supprimer un match dans les conditions suivantes :
\begin{itemize}
    \item Le match a commencé ou est terminé
    \item Un pari a été fait sur le match
\end{itemize}

3. Aucune indication sur les performances n'est indiquée dans le cahier des charges. Nous supposons que les performances
attendues pour les affichages de pages sont inférieures à 1 seconde. Nous supposons également que les performances attendues
pour les opérations de l'API sont inférieures à 0,5 seconde. À l'exception de l'opération de clôture d'un match qui nécessite
des calculs complexes.

4. Concernant le statut du match, nous supposons que le statut \textit{En cours} est défini par le fait que le commentateur a commencé à commenter le match.

\subsection{Mission principale du système}

\subsection{Environnement}

\chapter{Exigences fonctionnelles}

\begin{landscape}
  
  ...


\end{landscape}

\chapter{Exigences non fonctionnelles}


\chapter{Environnement de travail}

Nous avons choisi d'utiliser les outils suivants pour le développement de l'application:
\begin{itemize}
    \item \textbf{IDE} : VSCode
    \item \textbf{Langage} : PHP 8.2
    \item \textbf{Système} : Linux via WSL2
\end{itemize}

\chapter{Plan de test}

Les tests automatisés sont réalisés avec PHPUnit:

\begin{itemize}
    \item \textbf{Tests unitaires} : Tests des méthodes des classes
    \item \textbf{Tests fonctionnels} : Tests des routes de l'API
    \item \textbf{Tests d'intégration} : Tests des interactions entre les classes
    \item \textbf{Tests applicatif} : Tests de navigation sur l'application
\end{itemize}

Une couverture de test de 70\% est attendue pour les tests unitaires et fonctionnels.

TODO: chemin
TODO: lancement

Un test de performance réalisé avec Locust est également prévu. Il permettra de vérifier que les performances attendues sur les parties accessibles
à un public non connecté sont bien respectées (liste des rencontres par le site web et l'API).

TODO: caputre d'écran

\begin{appendix}
    \listoffigures
    \listoftables
  \end{appendix}
\end{document}
